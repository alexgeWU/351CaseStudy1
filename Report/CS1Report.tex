\documentclass[conference]{IEEEtran}
\usepackage{cite}
\usepackage{amsmath,amssymb,amsfonts}
\usepackage{algorithmic}
\usepackage{graphicx}
\usepackage{textcomp}
\usepackage{xcolor}
\usepackage{comment}
\def\BibTeX{{\rm B\kern-.05em{\sc i\kern-.025em b}\kern-.08em
    T\kern-.1667em\lower.7ex\hbox{E}\kern-.125emX}}
\begin{document}

\title{Modeling a Multi-Band Equalizer for Audio Enhancement and Spectral Analysis\\
}

\author{\IEEEauthorblockN{Samir Afsary}
\IEEEauthorblockA{\textit{Electrical and Systems Engineering} \\
\textit{Washington University in St. Louis}\\
St. Louis, MO \\
a.samir@wustl.edu}
\and
\IEEEauthorblockN{Ahmad Hamzeh}
\IEEEauthorblockA{\textit{Electrical and Systems Engineering} \\
\textit{Washington University in St. Louis}\\
St. Louis, MO \\
a.hamzeh@wustl.edu}
\and
\IEEEauthorblockN{Alex Ge}
\IEEEauthorblockA{\textit{Computer Engineering} \\
\textit{Washington University in St. Louis}\\
St. Louis, MO \\
age@wustl.edu}
}

\maketitle

\begin{abstract}
[insert text]
\end{abstract}


\section{Background}

Audio recordings often contain unwanted noise that can degrade the quality of the sound. 
The ability to selectively amplify or attenuate specific frequency bands is fundamental to modern audio processing systems, 
including equalizers found in smartphones, music players, and professional audio equipment.
By modifying the spectral content of a signal, equalization can enhance clarity, tonal balance, and supress unwanted noise.

A multi-band equalizer is a powerful tool for enhancing audio by allowing users to adjust the amplitude of specific frequency bands.
This is accomplished by decomposing an input signal into several frequency-selective components using filters
such as low-pass, high-pass, band-pass, and band-stop filters. 
Each band is independently scaled by a gain factor before recombining the filtered signals.
This structure enables targeted control over bass, midrange, and treble content while preserving overall signal integrity.

Frequency-selective filtering can be implemented using continuous-time RC circuit models, which provide well-characterized impulse and frequency responses. 
First-order RC filters allow straightforward analytical modeling of magnitude and phase behavior, making them suitable for constructing modular equalizer stages. 
When multiple filters operate in parallel and are combined through weighted summation, the resulting system achieves customizable spectral shaping.

Equalization additionally plays a critical role in environmental audio analysis.
Natural recordings, such as bird vocalizations, often include wind noise and broadband interference that 
obscure important biologically relevant features. Spectrogram analysis, the visualization of time-varying 
frequency content, enables identification of recurring vocalization patterns.
Through appropriate frequency bands and gain settings, an equalizer can isolate and amplify bird calls while reducing background noise,
facilitating more accurate species identification and behavioral studies.

This project focuses on modeling and designing a continuous-time multi-band equalizer to enhance music recordings and improve the clarity and 
identification of bird vocalizations in noisy environments through spectral analysis.





\begin{comment}
\begin{figure}[H]
\centerline{\includegraphics[width = 9cm, height = 7cm]{figures/Part1Plots/Controls2.png}}
\caption{With prey immigrants.}
\label{fig}
\end{figure}
\end{comment}



\section{Methods}
[insert text]


\section{Results}
[insert text]

\section{Conclusion}
[insert text]



\begin{thebibliography}{00}
\bibitem{b1} T. Tahara, M. K. A. Gavina, T. Kawano, and others, “Asymptotic stability of a modified Lotka-Volterra model with small immigrations,” Sci. Rep., vol. 8, p. 7029, May 2018. [Online]. Available: https://doi.org/10.1038/s41598-018-25436-2.=
\bibitem{b2} National Park Service, "The Challenge of Understanding Northern Yellowstone Elk Dynamics After Wolf Reintroduction," Yellowstone National Park, 2016. [Online]. Available: https://www.nps.gov/yell/learn/ys-24-1-the-challenge-of-understanding-northern-yellowstone-elk-dynamics-after-wolf-reintroduction.htm. [Accessed: Mar. 15, 2025].
\bibitem{b3} M. E. Gilpin, "The role of spatial structure in predator-prey dynamics," Ecology, vol. 89, no. 9, pp. 2585-2597, 2008. [Online]. Available: https://pubmed.ncbi.nlm.nih.gov/18707412/. [Accessed: Mar. 15, 2025].

\end{thebibliography}

\end{document}